\documentclass{beamer}
\usetheme{Berkeley}

\title[Beamer]{About the Beamer class in presentation making}

\subtitle{A short story}

\author{A.B.Arthur \and J.Doe}

\date{\today}
\institute{UCB}
\begin{document}

\begin{frame}
\titlepage
\begin{figure}
\centering
\includegraphics[scale=•]{•}
\end{figure}
\end{frame}


\begin{frame}
\frametitle{Table of Contents}
\tableofcontents
\end{frame}

\section{Motivation}
\subsection{The Basic Problem That We Studied}
\begin{frame}
\frametitle{Sample frame title}
This is some text in the first frame. This is some text in the first frame. This is some text in the first frame.
\end{frame}



\begin{frame}
\frametitle{What Are Prime Numbers?}

\begin{definition}
A \alert{prime number} is a number that has exactly two divisors.
\end{definition}

\begin{example}
\begin{itemize}
\item 2 is prime (two divisors: 1 and 2).
\pause
\item 3 is prime (two divisors: 1 and 3).
\pause
\item 4 is not prime (\alert{three} divisors: 1, 2, and 4).
\end{itemize}
\end{example}

\end{frame}

\section{Timepass}

\begin{frame}
\frametitle{There Is No Largest Prime Number}
\framesubtitle{The proof uses \textit{reductio ad absurdum}.}
\begin{theorem}
There is no largest prime number.
\end{theorem}
\begin{proof}
\begin{enumerate}
\item<1-> Suppose $p$ were the largest prime number.
\item<2-> Let $q$ be the product of the first $p$ numbers.
\item<3-> Then $q + 1$ is not divisible by any of them.
\item<1-> But $q + 1$ is greater than $1$, thus divisible by some prime
number not in the first $p$ numbers.\qedhere
\end{enumerate}
\end{proof}
\uncover<4->{The proof used \textit{reductio ad absurdum}.}
\end{frame}

\begin{frame}
\frametitle{What’s Still To Do?}
\begin{block}{Answered Questions}
How many primes are there?
\end{block}
\begin{block}{Open Questions}
Is every even number the sum of two primes?
\end{block}
\end{frame}


\begin{frame}
\frametitle{What’s Still To Do?}
\begin{itemize}
\item Answered Questions
\begin{itemize}
\item How many primes are there?
\end{itemize}
\item Open Questions
\begin{itemize}
\item Is every even number the sum of two primes?
\end{itemize}
\end{itemize}
\end{frame}


\begin{frame}
\frametitle{What’s Still To Do?}

\begin{columns}
\column{.5\textwidth}
\begin{block}{Answered Questions}
How many primes are there?
\end{block}

\column{.5\textwidth}
\begin{alertblock}{Open Questions}
Is every even number the sum of two primes?
\end{alertblock}
\begin{exampleblock}{Open Questions}
Is every even number the sum of two primes?
\end{exampleblock}

\end{columns}
\end{frame}


\begin{frame}{A title}
Some content.
\end{frame}


\begin{frame}[<+->]
\begin{theorem}
$A = B$.
\end{theorem}
\begin{proof}
\begin{itemize}
\item Clearly, $A = C$.
\item As shown earlier, $C = B$.
\item<3-> Thus $A = B$.
\end{itemize}
\end{proof}
\end{frame}

\begin{frame}[fragile]
\frametitle{An Algorithm For Finding Prime Numbers.}
\begin{verbatim}
int main (void)
{
std::vector<bool> is_prime (100, true);
for (int i = 2; i < 100; i++)
if (is_prime[i])
{
std::cout << i << " ";
for (int j = i; j < 100; is_prime [j] = false, j+=i);
}
return 0;
}
\end{verbatim}
\uncover<2->{Note the use of}
\end{frame}


\begin{frame}[<+->]{\texttt{\textless+-$\vert$alert@+\textgreater}}
\begin{itemize}
\item Robert De Niro
\item Brian De Palma
\item Gerard Depardieu
\item Tux
\end{itemize}
\end{frame}

\end{document}