\documentclass{article}
\usepackage{graphicx}
\usepackage{blindtext}
\usepackage{amsmath}
\usepackage{amsfonts}
\usepackage{amssymb}
\usepackage{multicol}
\usepackage{graphicx}
\usepackage[usenames,dvipsnames]{color}
\usepackage{xcolor}
\graphicspath{ {./} }
\usepackage{mathtools}
\begin{document}

\title{Introduction to \LaTeX{}}
\author{Tanishq Trivedi}
\date{\today}

\maketitle

\section{LAB 1}
The solutions to lab assignment 1

$$ 
C(n,r) = n!/r!(n-r)! 
$$

The following is in-line $ a + b = c - d = xy = w/z $ and following is displayed mathematical text 
$$
    a + b = c - d = xy = w/z
$$

\begin{equation}
    \label{simple equation}
    \alpha\beta = \gamma + \delta\textbf{}
\end{equation}

\begin{equation}
    \label{simple equation}
    \Gamma(n) = (n-1)!
\end{equation}


If $ Y_{rc} , r = 1,...,R,c = 1,...,C $ are random variables, show that


\section{Lists}
Here is the ...
\begin{enumerate}
    \item You can mix list environments as much as you like 
    \begin{itemize}
        \item But it might start to look silly
        \item [--] With different symbols
    \end{itemize}
    
    \item So do remember 
    \begin{description}
        \item \textbf{Stupid} things will not become smart because they are in a list
        \item \textbf{Smart} things, though, can be presented beautifully in a list 
    \end{description}
\end{enumerate}

\section{Mathematical symbols}

$$
\textrm {What are the points where }  \tfrac{\delta}{\delta x} f(x, y) = \tfrac{\delta}{\delta y} f(x, y) = 0 ? 
$$

$$
\nabla^{2}f(x, y) = \tfrac{\delta^{2}f}{\delta^{2}x} + \tfrac{\delta^{2}f}{\delta^{2}y}
$$

This is a matrix
\[
\begin{bmatrix}
aa & \cdots & az\\
\vdots & \ddots & \vdots\\
za & \cdots & zz
\end{bmatrix}
\]

\section{Other mathematical symbols}

\[ 
\int csc^{2}x \,dx = -cot x + C \hspace{2cm}
\lim_{\alpha\to 0} \tfrac{sin \alpha}{\alpha} = 1
\]
\[
h_i(t) = \lim_{\epsilon\to 0} \tfrac{1}{\epsilon}\tfrac{P(t<T_i\le t+\epsilon)}{P(T_i>t)}
\]

\section{Question 37}
The universe is immense and it seems to be homogeneous, 
in a large scale, everywhere we look at. \\
\includegraphics[width=\textwidth]{img-1}
There's a picture of a galaxy above

This example shows some instances of using the \texttt{xcolor} package 
to change the colour of elements in \LaTeX.

\begin{itemize}
\color{blue}
\item First item
\item Second item
\end{itemize}

\noindent
{\color{red} \rule{\linewidth}{0.5mm}}
\begin{itemize}
\color{ForestGreen}
\item First item
\item Second item
\end{itemize}

\noindent
{\color{RubineRed} \rule{\linewidth}{0.5mm}}

The background colour of text can also be \textcolor{red}{easily} set. For 
instance, you can change use an \colorbox{BurntOrange}{orange background} and then continue typing.

\LaTeX{} \cite{lamport94} is a set of macros built atop \TeX{} \cite{texbook}.

\begin{thebibliography}{9}
\bibitem{texbook}
Donald E. Knuth (1986) \emph{The \TeX{} Book}, Addison-Wesley Professional.

\bibitem{lamport94}
Leslie Lamport (1994) \emph{\LaTeX: a document preparation system}, Addison
Wesley, Massachusetts, 2nd ed.
\end{thebibliography}

\end{document}